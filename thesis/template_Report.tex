\documentclass[]{report}


% Title Page
\title{Machine Learning: The Hubbard model}
\author{Niels Billiet}


\begin{document}
\maketitle

\begin{abstract}
\end{abstract}

\tableofcontents

\section{Theoretical background: Quantumchemical fundamentals}

\subsection{Quantumchemical description of molecules}
\subsubsection{Hartree-Fock method}
\subsubsection{Full CI method}

\subsection{The Hubbard model}
\subsubsection{The Hubbard Hamiltonian}
\subsubsection{Overview of the PES of the simplest system}


\section{Theoretical background: Neural networks}

\subsection{Mathematical foundation of the network}
\subsubsection{General overview of neural networks}
\subsubsection{Activation functions}
\subsubsection{Optimization methods}
\subsection{Regularization methods}

\subsection{Network analysis}
\subsubsection{Training analysis}
\subsubsection{Weight analysis}
\subsubsection{Activation analysis}


\section{Research overview}
\subsection{Identification of the roadblocks in modern quantumchemical research}
\subsection{Solutions}
\subsection{Overview of planned research topics}


\section{Results with regard to predictive power of neural networks}

\subsection{Preliminary test: Can a simple network capture the data in a simple well defined system?}

\subsection{Simple hyperparameter sweep: Performance vs network compelexity}

\subsection{Data augmentation}
\subsubsection{Introduction of permutational symmetries in data}
\subsubsection{Introduction of noise on the output data}

\subsection{Performance of networks trained on less complex systems}

\subsection{Performance analysis on networks with simplest architectures}

\section{Results with regard to the learned features of the neural networks}

\subsection{PES of the neural network}

\subsection{Feature recognision}

\subsection{Stability analysis of the network}

\end{document}          
